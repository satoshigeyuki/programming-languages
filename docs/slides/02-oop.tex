\documentclass[12pt,dvipdfmx]{beamer}
\usepackage{graphicx}
\DeclareGraphicsExtensions{.pdf}
\DeclareGraphicsExtensions{.eps}
\graphicspath{{out/}{out/tex/}{out/tex/gpl/}{out/tex/svg/}{out/tex/lsvg/}{out/tex/dot/}}
% \graphicspath{{out/}{out/tex/}{out/pdf/}{out/eps/}{out/tex/gpl/}{out/tex/svg/}{out/pdf/dot/}{out/pdf/gpl/}{out/pdf/img/}{out/pdf/odg/}{out/pdf/svg/}{out/eps/dot/}{out/eps/gpl/}{out/eps/img/}{out/eps/odg/}{out/eps/svg/}}
\usepackage{listings}
\usepackage{fancybox}
\usepackage{hyperref}
\usepackage{color}

%%%%%%%%%%%%%%%%%%%%%%%%%%%
%%% themes
%%%%%%%%%%%%%%%%%%%%%%%%%%%
\usetheme{Szeged} % Szeged
%% no navigation bar
% default boxes Bergen Boadilla Madrid Pittsburgh Rochester
%% tree-like navigation bar
% Antibes JuanLesPins Montpellier
%% toc sidebar
% Berkeley PaloAlto Goettingen Marburg Hannover Berlin Ilmenau Dresden Darmstadt Frankfurt Singapore Szeged
%% Section and Subsection Tables
% Copenhagen Luebeck Malmoe Warsaw

%%%%%%%%%%%%%%%%%%%%%%%%%%%
%%% innerthemes
%%%%%%%%%%%%%%%%%%%%%%%%%%%
% \useinnertheme{circles}	% default circles rectangles rounded inmargin

%%%%%%%%%%%%%%%%%%%%%%%%%%%
%%% outerthemes
%%%%%%%%%%%%%%%%%%%%%%%%%%%
% outertheme
% \useoutertheme{default}	% default infolines miniframes smoothbars sidebar sprit shadow tree smoothtree


%%%%%%%%%%%%%%%%%%%%%%%%%%%
%%% colorthemes
%%%%%%%%%%%%%%%%%%%%%%%%%%%
\usecolortheme{seahorse}
%% special purpose
% default structure sidebartab 
%% complete 
% albatross beetle crane dove fly seagull 
%% inner
% lily orchid rose
%% outer
% whale seahorse dolphin

%%%%%%%%%%%%%%%%%%%%%%%%%%%
%%% fontthemes
%%%%%%%%%%%%%%%%%%%%%%%%%%%
\usefonttheme{serif}  
% default professionalfonts serif structurebold structureitalicserif structuresmallcapsserif

%%%%%%%%%%%%%%%%%%%%%%%%%%%
%%% generally useful beamer settings
%%%%%%%%%%%%%%%%%%%%%%%%%%%
% 
\AtBeginDvi{\special{pdf:tounicode EUC-UCS2}}
% do not show navigation
\setbeamertemplate{navigation symbols}{}
% show page numbers
\setbeamertemplate{footline}[frame number]

%%%%%%%%%%%%%%%%%%%%%%%%%%%
%%% define some colors for convenience
%%%%%%%%%%%%%%%%%%%%%%%%%%%

\newcommand{\mido}[1]{{\color{green}#1}}
\newcommand{\mura}[1]{{\color{purple}#1}}
\newcommand{\ore}[1]{{\color{orange}#1}}
\newcommand{\ao}[1]{{\color{blue}#1}}
\newcommand{\aka}[1]{{\color{red}#1}}

\setbeamercolor{ex}{bg=cyan!20!white}

%%%%%%%%%%%%%%%%%%%%%%%%%%%
%% customize beamer template
%% https://www.opt.mist.i.u-tokyo.ac.jp/~tasuku/beamer.html
%%%%%%%%%%%%%%%%%%%%%%%%%%%

\iffalse
%\renewcommand{\familydefault}{\sfdefault}  % 英文をサンセリフ体に
%\renewcommand{\kanjifamilydefault}{\gtdefault}  % 日本語をゴシック体に
\usefonttheme{structurebold} % タイトル部を太字
\setbeamerfont{alerted text}{series=\bfseries} % Alertを太字
\setbeamerfont{section in toc}{series=\mdseries} % 目次は太字にしない
\setbeamerfont{frametitle}{size=\Large} % フレームタイトル文字サイズ
\setbeamerfont{title}{size=\LARGE} % タイトル文字サイズ
\setbeamerfont{date}{size=\small}  % 日付文字サイズ

\definecolor{UniBlue}{RGB}{0,150,200} 
\definecolor{AlertOrange}{RGB}{255,76,0}
\definecolor{AlmostBlack}{RGB}{38,38,38}
\setbeamercolor{normal text}{fg=AlmostBlack}  % 本文カラー
\setbeamercolor{structure}{fg=UniBlue} % 見出しカラー
\setbeamercolor{block title}{fg=UniBlue!50!black} % ブロック部分タイトルカラー
\setbeamercolor{alerted text}{fg=AlertOrange} % \alert 文字カラー
\mode<beamer>{
    \definecolor{BackGroundGray}{RGB}{254,254,254}
    \setbeamercolor{background canvas}{bg=BackGroundGray} % スライドモードのみ背景をわずかにグレーにする
}


%フラットデザイン化
\setbeamertemplate{blocks}[rounded] % Blockの影を消す
\useinnertheme{circles} % 箇条書きをシンプルに
\setbeamertemplate{navigation symbols}{} % ナビゲーションシンボルを消す
\setbeamertemplate{footline}[frame number] % フッターはスライド番号のみ

%タイトルページ
\setbeamertemplate{title page}{%
    \vspace{2.5em}
    {\usebeamerfont{title} \usebeamercolor[fg]{title} \inserttitle \par}
    {\usebeamerfont{subtitle}\usebeamercolor[fg]{subtitle}\insertsubtitle \par}
    \vspace{1.5em}
    \begin{flushright}
        \usebeamerfont{author}\insertauthor\par
        \usebeamerfont{institute}\insertinstitute \par
        \vspace{3em}
        \usebeamerfont{date}\insertdate\par
        \usebeamercolor[fg]{titlegraphic}\inserttitlegraphic
    \end{flushright}
}
\fi

%%%%%%%%%%%%%%%%%%%%%%%%%%%
%%% how to typset code
%%%%%%%%%%%%%%%%%%%%%%%%%%%

\lstset{language = C,
numbers = left,
numberstyle = {\tiny \emph},
numbersep = 10pt,
breaklines = true,
breakindent = 40pt,
frame = tlRB,
frameround = ffft,
framesep = 3pt,
rulesep = 1pt,
rulecolor = {\color{blue}},
rulesepcolor = {\color{blue}},
flexiblecolumns = true,
keepspaces = true,
basicstyle = \ttfamily\scriptsize,
identifierstyle = ,
commentstyle = ,
stringstyle = ,
showstringspaces = false,
tabsize = 4,
escapechar=\@,
}

\title{Programming Lanaugages (2) \\
Essence of Object-Oriented Programming}
\institute{}
\author{Kenjiro Taura}
\date{}

\AtBeginSection[]
{
\begin{frame}
\frametitle{Contents}
\tableofcontents[currentsection]
\end{frame}
}

\iffalse
\AtBeginSubsection[]
{
\begin{frame}
\frametitle{Contents}
\tableofcontents[currentsection,currentsubsection]
\end{frame}
}
\fi

\begin{document}
\maketitle

%%%%%%%%%%%%%%%%%%%%%%%%%%%%%%%%%% 
% \begin{frame}
% \frametitle{Contents}
% \tableofcontents
% \end{frame}

%%%%%%%%%%%%%%%%% 
\begin{frame}[fragile]
  \frametitle{Classes and objects}

  \begin{itemize}
  \item a \ao{\it class} $\approx$ a data type definition $+$
    functions (\ao{\it methods}) for it
  \item an \ao{\it object} is a data instance created from a class definition
  \end{itemize}

  \begin{center}
\begin{lstlisting}[language=python]
# define a @\ao{\it class}@ named rect
class rect:
  def __init__(self, x, y, width, height):
    self.x = x
    self.y = y
    self.width = width
    self.height = height

r = rect(10,20,30,40) # create @\ao{\it an instance} (or \ao{\it an object}) of rect@
\end{lstlisting}
  \end{center}
\end{frame}

%%%%%%%%%%%%%%%%% 
\begin{frame}[fragile]
  \frametitle{Methods}
  \begin{itemize}
  \item $\approx$ functions
  \item unlike ordinary functions, a method of the same name can be defined
    for multiple classes (i.e., implemented differently)
  \end{itemize}

  \begin{center}
\begin{lstlisting}[language=python]
class rect:
  ...
  # define a @\ao{\it method}@ named area
  def area(self):
    return self.width * self.height

class ellipse:
  ...
  # define @\ao{\it another}@ method named area
  def area(self):
    return self.rx * self.ry * math.pi
  \end{lstlisting}
\end{center}
\end{frame}

%%%%%%%%%%%%%%%%% 
\begin{frame}[fragile]
  \frametitle{Dynamic dispatch}
  \begin{itemize}
  \item when you call a method, which method gets called among many implementations is determined by the class argument(s) belong to
  \end{itemize}

\begin{center}
\begin{lstlisting}[language=python]
# shapes may have both rect and ellipse instances
for s in shapes:
    ... s.area() ...
\end{lstlisting}
\end{center}
  
\end{frame}

%%%%%%%%%%%%%%%%% 
\begin{frame}[fragile]
  \frametitle{Language design points}
\begin{center}
\begin{lstlisting}[language=python]
# shapes may have both rect and ellipse instances
for s in shapes:
    ... s.area() ...
\end{lstlisting}
\end{center}
  
  \begin{itemize}
  \item in a code like the above, a variable {\tt s} may take a value of different classes (types) over time \ao{\it (polymorphism)}
  \item for languages that require type declarations,
    {\it how to declare/specify the type of {\tt s} or {\tt shapes}?}
  \item {\it does Go/Julia/OCaml/Rust require type declarations?}
  \end{itemize}
\end{frame}

%%%%%%%%%%%%%%%%% 
\begin{frame}[fragile]
  \frametitle{Language design points}
\begin{center}
\begin{lstlisting}[language=python]
# shapes may have both rect and ellipse instances
for s in shapes:
    ... s.area() ...
\end{lstlisting}
\end{center}
  \begin{itemize}
  \item more fundamentally, how can we guarantee, prior to execution,
      that \ao{\it type errors ($\approx$ application of non-existing methods)
      do not happen at runtime?}
  \item such property is called \ao{\it type safety}
  \item an algorithm that checks type safety
    prior to execution is often called
    \ao{\it static type checking}
    
  \item \ao{\it does Go/Julia/OCaml/Rust guarantee type safety?}
  \end{itemize}
\end{frame}

%%%%%%%%%%%%%%%%% 
\begin{frame}[fragile,allowframebreaks]
  \frametitle{Different approaches}

  \begin{enumerate}
  \item forgo static type checking and thus type safety
    (e.g., Python, javascript, Lisp, Smalltalk, \ldots)

\begin{lstlisting}[language=python]
shapes = [rect(...), ellipse(...), ...]
for s in shapes:
    ... s.area() ...
\end{lstlisting}
    
  \item disallow polymorphism altogether and make it (trivially) type-safe
    (e.g., Pascal)

\begin{lstlisting}[language=python]
rects : @{\ao{\tt array of rect}}@ = [ rect(...), rect(...) ]
for s : @{\ao{\tt rect}}@ in rects:
    ... s.area() ...
\end{lstlisting}
  \item do some (loose) static type checking without guaranteeing type safety;
    allow polymorphism via unsafe casts between pointers (e.g., C)

\begin{lstlisting}[language=python]
void * shapes[] = { (void *)rect(...), (void *)ellipse(...) };
for s in shapes:
    ... area(s) ...
\end{lstlisting}

  \item allow polymorphism yet guarantee type safety via \ao{\it subtypes}
    \begin{itemize}
    \item \ao{\it $C$ is a subtype of $P$ ($C \leq P$)} $\equiv$
      a value of $C$ can be safely used wherever $P$ is expected
    \item allow $P \leftarrow C$
      (put a value of type $C$ in a variable of type $P$)
    \end{itemize}
  \end{enumerate}
\end{frame}


%%%%%%%%%%%%%%%%% 
\begin{frame}
  \frametitle{Different approaches to subtyping}

  \begin{itemize}
  \item \ao{\it class} vs. \ao{\it interface}
    \begin{itemize}
    \item subtype relations hold between two \ao{\it classes}
    \item subtype relations hold between an \ao{\it interface} (or {\it trait, abstract class}, etc.) and a \ao{\it class} that {\it implements or conforms to} it; or between two \ao{\it interfaces}
    \end{itemize}
  \end{itemize}

  \begin{center}
    \includegraphics[width=0.35\textwidth]{out/pdf/svg/interface_2.pdf}
    \includegraphics[width=0.35\textwidth]{out/pdf/svg/interface_1.pdf}
  \end{center}

  \begin{itemize}
  \item \ao{\it nominal (explicit)} vs. \ao{\it structural} subtyping
    \begin{itemize}
    \item nominal : subtype relation exists only when so declared or a class is explicitly derived from the other
    \item structural : subtype relation exists whenever safe
      (based on the structure)
    \end{itemize}
  \end{itemize}
\end{frame}

%%%%%%%%%%%%%%%%% 
\begin{frame}
  \frametitle{How/if they guarantee type safety?}
  \begin{itemize}
  \item following slides briefly explain how Go/Rust/OCaml guarantee \ao{\it type safety}
  \item \ao{\it type safety} $\equiv$ ``no such methods'' error never happens at runtime
    $\equiv$ when a program containing $o.m(\ldots)$ passes static type check,
    $o$ always has method $m$ at runtime
  \item recall that this is not the case for some languages (including Python, Julia, C++, etc.)
  \end{itemize}
\end{frame}

%%%%%%%%%%%%%%%%% 
\begin{frame}
  \frametitle{A common framework}
  \begin{itemize}
  \item a type checker, {\it before execution}, 
    computes (or assumes given by the programmer)
    the \ao{\it static type} of each expression/variable
  \item for any \ao{\it assignment-like operations} $o = p$,
    it gets \ao{\it static types} of $o$ ($= S$) and $p$ ($= T$)
  \item the assignment is valid $\iff T \leq S$ 
  \end{itemize}
\end{frame}

\iffalse
%%%%%%%%%%%%%%%%% 
\begin{frame}
  \frametitle{A common framework}
  \begin{itemize}
  \item we (i.e., static type checker) like to guarantee that,
    \begin{itemize}
    \item for any expression $E$ whose \ao{\it static type} is $S$,
    \item any value $E$ could take at runtime can be \ao{\it safely put} in anywhere $S$ is expected
      ($\approx$ any such value implements all the methods $S$ specifies)
    \end{itemize}
  \item for which we have to guarantee that,
    for any \ao{\it assignment-like operations} $o = p$,
    where $o$'s \ao{\it static type} is $S$, 
    any value $p$ could take at runtime can be \ao{\it safely put}
    in anywhere $S$ is expected
  \item we want to check it by $p$'s static type ($T$) and $o$'s static type ($S$)
  \item this is precisely what we like to capture by
    \ao{\it subtype} relationship ($T \leq S$)
  \end{itemize}
\end{frame}
\fi

%%%%%%%%%%%%%%%%% 
\begin{frame}
  \frametitle{Note: assignment-like operations}
  \begin{itemize}
  \item  $\approx$ any operation in which the same value 
    changes its static type
    \begin{itemize}
    \item assignment to a variable/structure/array element
    \item function calls (passing values to parameters)
    \item function return (returning a value)
    \end{itemize}
  \end{itemize}
\end{frame}

%%%%%%%%%%%%%%%%% 
\begin{frame}
  \frametitle{Subtype relationship}
  \begin{itemize}
  \item $T$ is a subtype of $S$ ($T \leq S$)
  \item $\approx$
    any value of $T$ can be safely put anywhere $S$ is expected
  \item $\approx$
    \begin{enumerate}
    \item $T$ has all methods $S$ has 
    \item for each method,
      the input type of the $T$'s version is a \ao{\it supertype} of $S$'s
    \item for each method,
      the return type of the $T$'s version is a \ao{\it subtype} of $S$'s
    \end{enumerate}
  \item note: $P$ is a \ao{\it supertype} of $Q \iff Q \leq P$
    (i.e., $Q$ is a subtype of $P$)
  \end{itemize}
\end{frame}

%%%%%%%%%%%%%%%%% 
\begin{frame}
  \frametitle{Specifically, \ldots}
  \begin{itemize}
  \item imagine the type checker checks expression:
    \[ s.m(p) \]
    where
    \begin{itemize}
    \item $s$'s static type is $S$
    \item $S.m$'s input static type is $P$
    \item $S.m$'s return static type is $A$
    \end{itemize}
  \item and imagine $s$ is assigned a value $t$ ($s = t$) elsewhere,
    whose static type is $T$ 
  \item then
    \begin{itemize}
    \item $T$ must have $m$ (obvious)
    \item $T.m$'s input static type must be \ao{\it supertype} of $P$
    \item $T.m$'s return static type must be \ao{\it subtype} of $A$
    \end{itemize}
  \end{itemize}
\end{frame}

%%%%%%%%%%%%%%%%% 
\begin{frame}
  \frametitle{Go}
  \begin{itemize}
  \item details on \href{https://go.dev/ref/spec\#Assignability}{Assignability section of Go reference}
  \item<1-> Q: when is a type $T$ a subtype of another type $S$ ($T \leq S$)?
  \item<2-> A: one of the following
    \begin{enumerate}
    \item $S$ and $T$ are identical type
    \item $S$ is an interface and $T$ is a {\tt struct/interface} that {\it implements} $S$ or a pointer to it
    \end{enumerate}
  \item<3-> Q: so when is $T$ said to {\it implement} an interface $S$?
  \item<4-> A:
    \begin{itemize}
    \item $T$ has all the methods specified in $S$, and 
    \item each method in $T$ has the same type as the method of the same name in $S$
    \end{itemize}
  \end{itemize}
\end{frame}


%%%%%%%%%%%%%%%%% 
\begin{frame}
  \frametitle{Rust}
  \begin{itemize}
  \item<1-> Q: when is a type $T$ a subtype of another type $S$ ($T \leq S$)?
  \item<2-> A: one of the following
    \begin{enumerate}
    \item $S$ and $T$ are identical type
    \item $S$ is a reference to a \ao{\it trait}
      and $T$ is a reference to a {\tt struct} that {\it implements} $S$
    \end{enumerate}
  \item<3-> Q: so when does $T$ {\it implement} a trait  $S$?
  \item<4-> A:
    \begin{itemize}
    \item $T$ has all the methods specified in $S$, and 
    \item each method in $T$ has the same type as the method of the same name in $S$
    \end{itemize}
  \end{itemize}
\end{frame}

%%%%%%%%%%%%%%%%% 
\begin{frame}
  \frametitle{OCaml}
  \begin{itemize}
  \item<1-> Q: when is a type $T$ a subtype of another type $S$ ($T \leq S$)?
  \item<2-> A: one of the following, among others
    \begin{enumerate}
    \item<2-> $S$ and $T$ are identical type
    \item<3-> when each of $S$ and $T$ is a function type ($S = a \rightarrow b$
      and $T = a' \rightarrow b'$), then $b' \leq b$ and $a \leq a'$
    \item<4-> when each of $S$ and $T$ is an object type ($S = {\tt <} m_0 : t_0, \ldots {\tt >}$,
      $T = {\tt <} m'_0 : t'_0, \ldots {\tt >}$), then
      \begin{itemize}
      \item $\{ m_0, \ldots \} \subset \{ m'_0, \ldots \}$ and 
      \item for each $m_i = m'_j$, $t'_j \leq t_i$
      \end{itemize}
    \end{enumerate}
  \end{itemize}
\end{frame}

\end{document}



