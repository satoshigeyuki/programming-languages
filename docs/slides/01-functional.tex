\documentclass[12pt,dvipdfmx]{beamer}
\usepackage{graphicx}
\DeclareGraphicsExtensions{.pdf}
\DeclareGraphicsExtensions{.eps}
\graphicspath{{out/}{out/tex/}{out/tex/gpl/}{out/tex/svg/}{out/tex/lsvg/}{out/tex/dot/}}
% \graphicspath{{out/}{out/tex/}{out/pdf/}{out/eps/}{out/tex/gpl/}{out/tex/svg/}{out/pdf/dot/}{out/pdf/gpl/}{out/pdf/img/}{out/pdf/odg/}{out/pdf/svg/}{out/eps/dot/}{out/eps/gpl/}{out/eps/img/}{out/eps/odg/}{out/eps/svg/}}
\usepackage{listings}
\usepackage{fancybox}
\usepackage{hyperref}
\usepackage{color}

%%%%%%%%%%%%%%%%%%%%%%%%%%%
%%% themes
%%%%%%%%%%%%%%%%%%%%%%%%%%%
\usetheme{Szeged} % Szeged
%% no navigation bar
% default boxes Bergen Boadilla Madrid Pittsburgh Rochester
%% tree-like navigation bar
% Antibes JuanLesPins Montpellier
%% toc sidebar
% Berkeley PaloAlto Goettingen Marburg Hannover Berlin Ilmenau Dresden Darmstadt Frankfurt Singapore Szeged
%% Section and Subsection Tables
% Copenhagen Luebeck Malmoe Warsaw

%%%%%%%%%%%%%%%%%%%%%%%%%%%
%%% innerthemes
%%%%%%%%%%%%%%%%%%%%%%%%%%%
% \useinnertheme{circles}	% default circles rectangles rounded inmargin

%%%%%%%%%%%%%%%%%%%%%%%%%%%
%%% outerthemes
%%%%%%%%%%%%%%%%%%%%%%%%%%%
% outertheme
% \useoutertheme{default}	% default infolines miniframes smoothbars sidebar sprit shadow tree smoothtree


%%%%%%%%%%%%%%%%%%%%%%%%%%%
%%% colorthemes
%%%%%%%%%%%%%%%%%%%%%%%%%%%
\usecolortheme{seahorse}
%% special purpose
% default structure sidebartab 
%% complete 
% albatross beetle crane dove fly seagull 
%% inner
% lily orchid rose
%% outer
% whale seahorse dolphin

%%%%%%%%%%%%%%%%%%%%%%%%%%%
%%% fontthemes
%%%%%%%%%%%%%%%%%%%%%%%%%%%
\usefonttheme{serif}  
% default professionalfonts serif structurebold structureitalicserif structuresmallcapsserif

%%%%%%%%%%%%%%%%%%%%%%%%%%%
%%% generally useful beamer settings
%%%%%%%%%%%%%%%%%%%%%%%%%%%
% 
\AtBeginDvi{\special{pdf:tounicode EUC-UCS2}}
% do not show navigation
\setbeamertemplate{navigation symbols}{}
% show page numbers
\setbeamertemplate{footline}[frame number]

%%%%%%%%%%%%%%%%%%%%%%%%%%%
%%% define some colors for convenience
%%%%%%%%%%%%%%%%%%%%%%%%%%%

\newcommand{\mido}[1]{{\color{green}#1}}
\newcommand{\mura}[1]{{\color{purple}#1}}
\newcommand{\ore}[1]{{\color{orange}#1}}
\newcommand{\ao}[1]{{\color{blue}#1}}
\newcommand{\aka}[1]{{\color{red}#1}}

\setbeamercolor{ex}{bg=cyan!20!white}

%%%%%%%%%%%%%%%%%%%%%%%%%%%
%% customize beamer template
%% https://www.opt.mist.i.u-tokyo.ac.jp/~tasuku/beamer.html
%%%%%%%%%%%%%%%%%%%%%%%%%%%

\iffalse
%\renewcommand{\familydefault}{\sfdefault}  % 英文をサンセリフ体に
%\renewcommand{\kanjifamilydefault}{\gtdefault}  % 日本語をゴシック体に
\usefonttheme{structurebold} % タイトル部を太字
\setbeamerfont{alerted text}{series=\bfseries} % Alertを太字
\setbeamerfont{section in toc}{series=\mdseries} % 目次は太字にしない
\setbeamerfont{frametitle}{size=\Large} % フレームタイトル文字サイズ
\setbeamerfont{title}{size=\LARGE} % タイトル文字サイズ
\setbeamerfont{date}{size=\small}  % 日付文字サイズ

\definecolor{UniBlue}{RGB}{0,150,200} 
\definecolor{AlertOrange}{RGB}{255,76,0}
\definecolor{AlmostBlack}{RGB}{38,38,38}
\setbeamercolor{normal text}{fg=AlmostBlack}  % 本文カラー
\setbeamercolor{structure}{fg=UniBlue} % 見出しカラー
\setbeamercolor{block title}{fg=UniBlue!50!black} % ブロック部分タイトルカラー
\setbeamercolor{alerted text}{fg=AlertOrange} % \alert 文字カラー
\mode<beamer>{
    \definecolor{BackGroundGray}{RGB}{254,254,254}
    \setbeamercolor{background canvas}{bg=BackGroundGray} % スライドモードのみ背景をわずかにグレーにする
}


%フラットデザイン化
\setbeamertemplate{blocks}[rounded] % Blockの影を消す
\useinnertheme{circles} % 箇条書きをシンプルに
\setbeamertemplate{navigation symbols}{} % ナビゲーションシンボルを消す
\setbeamertemplate{footline}[frame number] % フッターはスライド番号のみ

%タイトルページ
\setbeamertemplate{title page}{%
    \vspace{2.5em}
    {\usebeamerfont{title} \usebeamercolor[fg]{title} \inserttitle \par}
    {\usebeamerfont{subtitle}\usebeamercolor[fg]{subtitle}\insertsubtitle \par}
    \vspace{1.5em}
    \begin{flushright}
        \usebeamerfont{author}\insertauthor\par
        \usebeamerfont{institute}\insertinstitute \par
        \vspace{3em}
        \usebeamerfont{date}\insertdate\par
        \usebeamercolor[fg]{titlegraphic}\inserttitlegraphic
    \end{flushright}
}
\fi

%%%%%%%%%%%%%%%%%%%%%%%%%%%
%%% how to typset code
%%%%%%%%%%%%%%%%%%%%%%%%%%%

\lstset{language = C,
numbers = left,
numberstyle = {\tiny \emph},
numbersep = 10pt,
breaklines = true,
breakindent = 40pt,
frame = tlRB,
frameround = ffft,
framesep = 3pt,
rulesep = 1pt,
rulecolor = {\color{blue}},
rulesepcolor = {\color{blue}},
flexiblecolumns = true,
keepspaces = true,
basicstyle = \ttfamily\scriptsize,
identifierstyle = ,
commentstyle = ,
stringstyle = ,
showstringspaces = false,
tabsize = 4,
escapechar=\@,
}

\title{Programming Lanaugages (1) \\
Functional Programming}
\institute{}
\author{Kenjiro Taura}
\date{}

\AtBeginSection[]
{
\begin{frame}
\frametitle{Contents}
\tableofcontents[currentsection]
\end{frame}
}

\iffalse
\AtBeginSubsection[]
{
\begin{frame}
\frametitle{Contents}
\tableofcontents[currentsection,currentsubsection]
\end{frame}
}
\fi

\begin{document}
\maketitle

%%%%%%%%%%%%%%%%%%%%%%%%%%%%%%%%%% 
% \begin{frame}
% \frametitle{Contents}
% \tableofcontents
% \end{frame}

%%%%%%%%%%%%%%%%% 
\begin{frame}
  \frametitle{What is functional programming? 1st attempt}
  \begin{itemize}
  \item Wikipedia:
    \begin{quote}
    \ldots functional programming is a programming paradigm where programs are constructed by applying and composing functions.
    \end{quote}

  \item \ldots, well, I am almost always writing and applying functions, \ldots
  \end{itemize}
\end{frame}

%%%%%%%%%%%%%%%%% 
\begin{frame}
  \frametitle{What is functional programming? 2nd attempt}
  \begin{itemize}
  \item ask ChatGPT: describe the most important aspect of functional programming in less than 100 words.
    \begin{quote}
      Functional programming's essence lies in treating computation as
      the evaluation of mathematical functions. It emphasizes
      \ao{\it immutability,} pure functions, and higher-order functions,
      enabling clearer, more concise, and maintainable code. By
      \ao{\it avoiding mutable state and side effects,} it enhances modularity
      and facilitates reasoning about program behavior.
    \end{quote}
  \item is it about \ao{\it ``avoiding side effects''}, or more generally
    \ao{\it avoiding certain tools that you have been taught and are now familiar with}?
  \item if so, why you care?
  \end{itemize}
\end{frame}

%%%%%%%%%%%%%%%%% 
\begin{frame}
  \frametitle{What is functional programming? My take}
  \begin{itemize}
  \item it is a ``way of thinking'' when you solve a problem
  \item it formulates the solution to the problem using \ao{\it recursion}
  \item let's dig into it using an example
  \end{itemize}
\end{frame}

%%%%%%%%%%%%%%%%% 
\begin{frame}[fragile]
  \frametitle{An example}
  \begin{itemize}
  \item Q: write a function that computes the sum of elements in an array
  \item A: a ``procedural'' Python program
\begin{lstlisting}
def sum_array(a):
    n = len(a)
    s = 0
    for i in range(n):
        s = s + a[i]
    return s
\end{lstlisting}
  \end{itemize}
\end{frame}

%%%%%%%%%%%%%%%%% 
\begin{frame}[fragile]
  \frametitle{Thinking behind the procedural version}
  \begin{itemize}
  \item well, to compute {\tt a[0] + a[1] + \ldots + a[n-1]}, 
  \item start with {\tt s = 0}, and
  \item {\tt s = s + a[0]}
  \item {\tt s = s + a[1]}
  \item \ldots
  \item {\tt s = s + a[n-1]}
  \item now {\tt s} should hold what we want
  \end{itemize}
remember how you overcame the following confusing ``equation''?
\begin{lstlisting}
s = s + a[i]  # do you mean 0 = a[i] ??
\end{lstlisting}
\end{frame}

%%%%%%%%%%%%%%%%% 
\begin{frame}[fragile]
  \frametitle{A ``functional'' version}
  \begin{itemize}
  \item []
\begin{lstlisting}
# a[i] + a[i+1] + ... + a[j-1]
def sum_range(a, i, j):
    if i == j:
        return 0
    else:
        return a[i] + sum_range(a, i + 1, j)

def sum_array(a):
    return sum_range(a, 0, len(a))
\end{lstlisting}
\end{itemize}
\end{frame}

%%%%%%%%%%%%%%%%% 
\begin{frame}[fragile]
  \frametitle{A (superficial) characteristics of the ``functional'' version}
  \begin{itemize}
  \item {\it no updates} to variables (like {\tt s = s + ...})
  \item no loops
  \end{itemize}
  but the point is not about \ao{\it avoiding} them
\end{frame}

%%%%%%%%%%%%%%%%% 
\begin{frame}[fragile]
  \frametitle{The thinking behind the functional version}
  \begin{itemize}
  \item the observation
    \[ \mbox{sum of {\tt a[0:n]}} = \mbox{\tt a[0]} + \mbox{sum of {\tt a[1:n]}} \]
  \item \ldots and you can compute ``sum of {\tt a[1:n]}'' (almost) by a recursive call
  \item to be precise, you define a function to compute sum of an array range {\tt a[i:j]} by
    \[ \mbox{sum of {\tt a[i:j]}} = \mbox{\tt a[i]} + \mbox{sum of {\tt a[i+1:j]}} \]
  \item one more thing is the base case (when {\tt i} $=$ {\tt j}, sum is zero)
  \end{itemize}
\end{frame}

%%%%%%%%%%%%%%%%% 
\begin{frame}[fragile]
  \frametitle{Note : a few more alternatives}
  \begin{itemize}
  \item 
    \[ \mbox{sum of {\tt a[i:j]}} = \mbox{sum of {\tt a[i:j-1]}} + \mbox{\tt a[j-1]} \]
  \item 
    \[ \mbox{sum of {\tt a[i:j]}} = \mbox{sum of {\tt a[i:c]}} + \mbox{\tt a[c:j]} \]
    where ${\tt c} = \left\lfloor ({\tt i} + {\tt j}) / 2 \right\rfloor$,
    or any value that satisfies ${\tt i} < {\tt c} < {\tt j}$, for that matter

\begin{lstlisting}
def sum_range(a, i, j):
    if i == j:
        return 0
    elif i + 1 == j:
        return a[i]
    else:
        c = (i + j) // 2
        return sum_range(a, i, c) + sum_range(a, c, j)
\end{lstlisting}
  \end{itemize}
\end{frame}

%%%%%%%%%%%%%%%%% 
\begin{frame}
  \frametitle{The ``functional way'' of problem solving}
  \begin{itemize}
  \item $\approx$ solving a problem by recursive calls
  \item $\approx$ solving a problem by assuming solutions to ``smaller'' cases are known
  \end{itemize}

  this is very powerful because of the same reason why solving math problems
  using \ao{\it recurrence relation (漸化式)} is very powerful
\end{frame}

%%%%%%%%%%%%%%%%% 
\begin{frame}
  \frametitle{Solving problems with recurrence relation : an example}
  \begin{itemize}
  \item Q: Draw $n$ lines in a plane,
    in such a way that no three lines intersect at a point.
    How many regions do they divide the plane into?
  \item A: Let the number of regions $a_n$. Then,
    \begin{columns}
      \begin{column}{0.5\textwidth}
    \[ \only<3>{\left\{}
        \begin{array}{rcl}
          \only<2->{a_0 & = & 1,} \\
          \only<3->{a_n & = & a_{n-1} + n} \\
        \end{array}
      \only<3>{\right.} \]
      \end{column}
      \begin{column}{0.5\textwidth}
        \only<1-2>{\includegraphics[width=\textwidth]{out/pdf/svg/recurrence_1.pdf}}%
        \only<3>{\includegraphics[width=\textwidth]{out/pdf/svg/recurrence_2.pdf}}
      \end{column}
    \end{columns}
  \end{itemize}
\end{frame}

%%%%%%%%%%%%%%%%% 
\begin{frame}
  \frametitle{The functional thinking}
  \begin{enumerate}
  \item say you are asked to find an answer to a problem (e.g., $f(n)$ or $g(a)$)
  \item try to answer it,
    {\it assuming the answer to \ao{``smaller cases''}
    are known}
  \item express it using recursions
  \end{enumerate}
  
  \begin{itemize}
  \item what \ao{\it ``smaller''} exactly means depends on the problem
    \begin{itemize}
    \item smaller integers (e.g., $n - 1$, $n/2$, etc.)
    \item smaller arrays (e.g., $a[0:n-1]$, $a[1:n]$, $a[0:n/2]$,
      numbers in $a$ less than $x$, etc.)
    \item a child of a tree node
    \item etc.
    \end{itemize}
  \end{itemize}
\end{frame}

%%%%%%%%%%%%%%%%% 
\begin{frame}
  \frametitle{The divide-and-conquer paradigm}
  \begin{itemize}
  \item a similarly powerful paradigm
    is the ``divide-and-conquer'' problem solving
  \item given an input $X$
  \item somehow \ao{\it ``divide''} $X$ into smaller
    instances $X_0, X_1, \cdots$
  \item solve each of them using a recursion
  \item somehow \ao{\it ``merge''} them into the solution to $X$
  \end{itemize}
\end{frame}

%%%%%%%%%%%%%%%%% 
\begin{frame}[fragile]
  \frametitle{One more example}
  \begin{itemize}
  \item Q: define a function that, given $a$ and $n$, computes $a^n$
    \begin{itemize}
    \item note: Python has a builtin primitive ($a$ {\tt **} $n$) or {\tt pow}
      that just does that, but here we define it without them
    \end{itemize}
  \item A: the ``procedural'' version
\begin{lstlisting}
def pow(a, n):
    p = 1
    for i in range(n):
        p = p * a
    return p
\end{lstlisting}
  \item this expresses \ao{\it how} you compute $a^n$, step by step
  \end{itemize}
\end{frame}

%%%%%%%%%%%%%%%%% 
\begin{frame}[fragile]
  \frametitle{A functional version}
  \begin{itemize}
  \item instead ask ``what is'' $a^n$
  \item well,
    \begin{itemize}
    \item base case: $n = 0 \Rightarrow 1$
    \item otherwise, $a^n = a * a^{n-1}$
\begin{lstlisting}
def pow(a, n):
    if n == 0:
        return 1
    else:
        return a * pow(a, n - 1)
\end{lstlisting}
    \end{itemize}
  \end{itemize}
\end{frame}  

%%%%%%%%%%%%%%%%% 
\begin{frame}[fragile]
  \frametitle{A smarter version}
\begin{lstlisting}
def pow(a, n):
    if n == 0:
        return 1
    elif n % 2 == 0:
        p = pow(a, n // 2)
        return p * p
    else:
        return a * pow(a, n - 1)
\end{lstlisting}
\end{frame}  


\end{document}


